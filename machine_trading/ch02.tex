%%
%% Author: abdul
%% 7/25/18
%%

% Preamble
\documentclass[11pt]{article}
\title{Factor Models}
\date{}

% Packages
\usepackage{a4wide}
\usepackage{amsmath}
\usepackage{hyperref}


% Document
\begin{document}
    \maketitle
    \begin{itemize}
        \item The reason some of us are reluctant to do hold a stock index for long(30 to 40 years)is market risk: buying-and-holding
        the stock index has long periods of drawdowns.
        \item What if we instead hold a long-short market-neutral portfolio? why not just long value stocks and short growth stocks?
        (we may hear that value stocks offer better long-term returns than growth stocks, value investor Warren Buffett)
        \item  Just because this portfolio doesn't have market risk doesn't mean it doesn't have some other risk associated with long value and short growth.
        \item many of the strategies whether they trade stocks or options, have such so-called factor risks, eg: volatility risk factor in option strategies
        \item the factor risk of each stock is both serially and cross-sectionally correlated over a period of time (? verify if this is for all strategies or for
        those in the above portfolio)
        \item ``alphas" - returns that are not known to be associated with factor risks.
        \item Earning alpha is almost riskless arbitrage, because their risks can be reduced to zero by increasing diversification.
        The more stocks one includes in a portfolio with positive alpha, the smaller the (white noise) risk compared to its return.
        \item On the other hand, factor risk cannot be diversified away. No matter how many value stocks you add to the long side of a portfolio,
        and how many growth stocks you add to the short side, the factor risk remains unchanged.
        \item alpha's, are especially hard to find. Every time someone finds alpha, they have to keep it a secret. And still, more and more people will discover it eventually,
        and the alpha diminishes.
        \item Even if nobody else finds it, alpha has limited capacity, so eventually it diminishes if too much capital has been applied to it
        \item most factor returns remain alive and well, year after year; trading factor models
        \item It is true that some factor returns do decrease over time - SMB factor, the put-call implied volatilities factor
        \item implicit factor exposures in strategies
        \item \textit{factor models}, This term is confined to simple linear models, where the predictors in the linear regression are our factors.
        \item regression coefficients for factors are sometimes called \textit{smart betas}, a ``dumb" beta being the regression coefficient between the return of a
        stock versus the return of the market index.
        \item difference between time-series and cross-sectional factors
    \end{itemize}
    \section*{Time-series Factors}
        \begin{itemize}
	            \item examples of time-series factors - the market return and the value-minus-growth return. they vary from time to time, but they do not vary across
            different stocks or financial assets
            \item The value-minus-growth factor is often called HML, because it is the returns of a basket of high book-to-market stocks minus that of a basket of
            low book-to-market stocks. Such a long-short portfolio whose returns represent a time-series factor is called a ``hedge" portfolio
            \item What do vary across different stocks are their returns in response to this factor. This response is called the \textit{factor loading (or factor exposure)},
            which is just a regression coefficient
            \item Time-series factors usually (but not always) have dimensions of returns (i.e., dollar divided by time), but factor loadings are dimensionless
           \begin{equation}
               Return(t, s) - r_F = \alpha(t,s) + \beta_1(s) * Factor_1(t) + \beta_2(s) * Factor_2(t) + \dots + \epsilon(t, s)
               \tag{2.1}\label{eq:2.1}
           \end{equation}
            \begin{flalign*}
                & Return(t, s) \text{ is return of a stock $s$ over a period from $t-1$ to $t$} \\
                & \beta_i(s) \text{ are the factor loadings of stock $s$} \\
            \end{flalign*}

            \item The $\alpha$ of a stock is included, indicating stock will have a time-varying return not driven by common factors.
            In our modeling in this chapter, we won't have any such alpha models, hence $\alpha$ will be set as a constant in time:
            $\alpha(t, s) = \alpha(s)$
            \item noise term $\epsilon$, a catchall term that captures anything that our factor or alpha models are unable to explain and is supposed to be uncorrelated
                both serially (in time) and cross-sectionally (across stocks)
            \item Since the time-series factors are observable (e.g., we can easily measure the market return) and known historically, all we need to do is use the usual
                least square fit to find the $\beta_i(s)$'s
            \item how do we use equation \ref{eq:2.1} for our investment decisions, the time index $t$ is the same on both sides of the equation,
                This means that this factor model are descriptive and explanatory, but not predictive
            \item turning equation \ref{eq:2.1} into a predictive equation
            \begin{flalign*}
                Return(t+1, s) - r_F = \alpha(s) + \beta_1(s) * Factor_1(t) + \beta_2(s) * Factor_2(t) + \dots + \epsilon(t, s)
                \tag{2.2}\label{eq:2.2}
            \end{flalign*}
            \item run a least square fit on equation $\ref{eq:2.2}$ and once we have obtained estimates of the $\alpha$ and $\beta_i$ for each stock,
                we can use equation $\ref{eq:2.2}$ for predicting the next period's return
            \item Besides the market return and HML, other popular time-series factors are SMB-the returns of a basket of small-market-capitalization stocks minus
                that of a basket of big-market-capitalization stocks, and UMD-the returns of a basket of stocks whose prices went up minus that of a basket of stocks
                that went down. (People also call this the WML factor: ``Winners minus Losers.")
            \item Interestingly, both the HML value factor and the UMD momentum factor have positive returns.
            \item Other macroeconomic factors include economic growth (specifically, real quarterly GDP growth and quarterly real personal consumption expenditures growth),
                quarterly Consumer Price Index change, and volatility (VIX) change.
            \item Example: Using Fama-French factors to predict the next-day return
                \subitem The three time-series factors discussed above, the excess market return (i.e., market return minus the risk free rate), the HML portfolio return,
                and the SMB portfolio return, are called the FamaFrench factors (Fama and French, 1993).
                \subitem They are usually used as contemporaneous factors to account for the current returns of a stock, here we will see whether they can be used as predictor factors to predict the next
                day's returns of stocks in the S\&P 500 Index
                \subitem For our backtest, we use mid-quotes at market close provided by the Center for Research of Security Prices (CRSP.com) survivorship-bias-free database from January 3, 2007, to December 31, 2013.
                \subitem We choose the midprice (half of the bid and ask prices) at the market close to represent the closing price in order to avoid effects of the widened bid-ask spread at the close.
                \subitem The daily market, HML, and SMB factors are from Professor French's website \url{http://mba.tuck.dartmouth.edu/pages/faculty/ken.french/data_library.html#Research}
                \subitem We divide this data into two halves, and use the first half as a trainset to estimate the factor loadings for each stock separately (i.e., in a multivariate, multiple regression).
                \subitem excess return (return minus risk free rate) for the next trading day and the predicted return for the next trading day
                \subitem We then buy the top 50 stocks with the largest predicted returns and short the bottom 50 with the smallest
                \subitem Since our data set is a survivorship-bias-free one, there are symbols that stopped trading as well as symbols that just started trading in the middle of the time period. Hence we can only make
                predictions for the subset of data that has daily returns on a certain day.
                \subitem the model works very well in-sample: it has a CAGR of 242 percent and a Sharpe ratio of 3.7
                \subitem However, it generated negative returns out-of-sample. Clearly, the Fama-French factors are not terribly useful for short-term predictions
        \end{itemize}

    \section*{Cross-sectional Factors}

    \section*{A Two-Factor Model}

    \section*{Short  Interest}

    \section*{Liquidity}

    \section*{Statistical Factors}

    \section*{Putting Them All Together}



\end{document}