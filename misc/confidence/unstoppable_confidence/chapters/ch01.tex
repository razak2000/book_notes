\chapter*{Introduction To Unstoppable Confidence}
% Document
\begin{document}
    \begin{itemize}
        \item Much of the material yet not all is based upon Neuro-Linguistic Programming (NLP).
        NLP is the study of how language, both verbal and nonverbal, affects our minds. By consciously
        directing our minds, we can create resourceful ways of behaving for ourselves.
        \item One of the NLP presuppositions is that since we all share the same neurology, that means whatever
        anyone can do, you can do as well provided that you direct your mind in exactly the same way. That
        means if something is possible for others, it’s possible for you too. If confidence is possible for
        others (and it definitely is), it is definitely possible for you too.
        \item A ll our dreams can come true, if we have the courage to pursue them. – Walt Disney
        \item deciding to do something and doing something are two entirely different things.
        \item Cathy took action and fulfilled her dreams. Samantha did not. The difference between the two is
        that Cathy had the confidence to put her plan into motion by taking action. Samantha lacked the
        confidence.
        \item Confidence means different things to different
        people. Similarly, confidence evokes certain feelings and automatic reactions within people.
        \item Arrogance is sometimes mistaken for confidence. Arrogance is something completely different from
        confidence. Arrogance is the notion of being somewhat confident with a leaning toward elitism,
        bragging, being macho, showing off, etc.
        \item True confidence
        comes from within and when you realize you are confident, you do not need to proclaim it to the
        world.
        \item The truly
        confident people all had a similar trait among them: their confidence came from within and it did not
        have to be voiced. A nonchalant, matter-of-fact confidence is ideal. There is no need to brag about
        one’s accomplishments. Those who brag are only masking their insecurity about themselves.
        \item Actions speak louder than words so take your confidence and
        make it happen.
        \item If someone has to continually broadcast his or her “confidence”, it really makes me wonder. It
        seems that they are not really that confident at all and instead they are often trying to convince
        themselves by telling others.
        \item Confidence comes from within and when you believe in yourself, others will believe in you. This is
        a universal law.
        \item Belligerence is what makes a jerk just that. The jerk is abrasive and cares little for the relationships
        he forms with others, but bulldozes his way through life. True confidence, however, allows you to
        go through life easily, to efficiently get the results you want and make people feel good when you
        deal with them.
        \item Make people feel good. Do this because you can.
        \item Confidence is vital because it is the difference that
        makes the difference. When people consistently take action and make the appropriate course
        corrections, they get massive result and achieve all their goals. However, if they lack that same
        confidence, they will stay stuck. It would be no different if they had no dreams or goals at all.
        \item Having the confidence, especially in the context of having the ability to communicate, is absolutely
        essential. Without it, people don’t communicate effectively. To the degree that you are confident
        and communicate well with others is the degree to which you will succeed in life.
        \item Are confidence and competence the same thing? No! In fact, they are quite distinct. Understanding
        the difference is essential.Competence: The ability to do something. Confidence: Your belief about your competence.
        \item Each of us has individual experiences, beliefs, and values that make us perceive life in our own
        individual way. Everyone has their own perception of reality, a unique model of the world. This
        means that it is only a perception and not actual reality. The empowering part of this idea is that
        these beliefs can be changed and thereby change one’s perception of life.
        \item People are an absolute recipe for disaster if they lack competence and yet have unjustified
        confidence.
        \item If someone is competent and yet lacks confidence, they will be stuck. They might have a perfect
        understanding of some powerful concepts yet they never take action. Someone who has the
        knowledge and competence to something and does nothing is no better off than someone who is
        clueless and incompetent. The proper fruit of knowledge is action.
        \item There are three key beliefs that will enable you to escape being stuck, and allow you to easily take
        action: You do not need to know everything to get started, You are a very resourceful person, If you do not know the answer to something, you can find the answer or the person who does
        know.
        \item You are free to know your piece of the puzzle
        while you let everyone else fill in their roles to accomplish what you want.
        \item Since you realize the difference between
        a stuck attitude and an empowering attitude, you will naturally become aware of any times that you
        might be acting stuck. At that point, back up and change your attitude by remembering your
        resourcefulness.
        \item Being unstoppable means having the competence and the confidence and going for it.
        \item When you have the competence to do what you want and the confidence to follow through and take
        action, you are unstoppable! This is the pinnacle of success. Nothing can nor will stop you from achieving what you want.
        \item You
        realize that anything in your path that seems like an obstacle or a detour really is an opportunity in
        disguise to show your resourcefulness.
        \item Here’s the simple success formula in a nutshell: You set an outcome. You monitor your results you get through feedback. You vary your behavior until you get your outcome.
        \item Knowing your outcome means having a vision of what you do want for yourself in your life. When
        you know what you want, you can get it much more easily. It’s impossible to get what you want if
        you haven’t first decided what you do in fact want.
        \item Monitoring your results through feedback means having the sensory acuity to know is working for
        you and what is not working for you. When you find something that works for you, you naturally do
        more of it and in an even better way if you can. When you find something that is not working, you
        ask yourself what you can do differently to get your results.
        \item What it means to vary your behavior until you get your outcome is that you persist and keep doing
        different things, operating with different strategies, and using different techniques until you finally
        achieve your outcome.
        \item An outcome is something you desire and/or want. It is
        similar to a goal yet can be much smaller. The acronym
        SMART stands for: SPECIFIC, MEASURABLE, ACHIEVABLE, REALISTIC, and TIMED.
        \item You make an outcome specific when you state what you will see, hear, feel and experience, things
        which will verify that you have undoubtedly achieved it.
        \item Your outcome is measurable when you have a clear way to know if you have met your outcome or
        not. While confidence is sometimes difficult to measure, you can get
        creative and find ways.
        \item Ensure that your outcome is achievable, that it is physically viable for you to accomplish. Ensure
        that you have a good likelihood of success. Remember that while you are unstoppable and that you can achieve anything that you desire, you must simultaneously plan a smooth progression, a driving
        movement from point A to point B that stretches your comfort zone as you march towards success.
        \item Avoid setting yourself up for frustration by setting an outcome that is unattainable in the short term.
        Plan your progression; know the small steps along the way that will lead you to your ultimate goals.
        Set achievable goals, achieve them, and reset your goals even higher!
        \item Realistic outcomes are outcomes that are based in reality. Asking for an unrealistic outcome only sets one up for failure.
        \item When you set
        realistic outcomes, you will be proud of yourself when you achieve them. Realistic means that it is
        something that can be done, even if in theory.
        \item As long as your outcome has a basis in reality or
        theoretical feasibility, it is realistic.
        \item Make sure your outcome is timed by attaching a specific deadline to its accomplishment.
        \item dreams are unlikely to
        be achieved unless there is a deadline attached along with a workable plan on how to achieve them.
        \item Goals are dreams with deadlines. Impotent goals do nothing for one’s motivation.
        \item A key to achieving SMART outcomes is to monitor your progress. Be accountable to yourself and
        check in periodically, to notice your results as consistently and frequently as you wish.
        \item No one
        needs to give you permission to accomplish all of your outcomes, achieve all of your goals, and live
        your dreams. You can do that right now!
        \item Each time you get the results you want, stop for a moment
        and congratulate yourself. Treat yourself somehow to something extraordinary. If you find that you
        are not yet getting the results you want, step back and look at what you can do differently. Keep
        doing things differently, adjusting your path until you get what you want.
        \item Gaining more confidence is a process rather than a single event. For each different area of our lives,
        we can expand our comfort zones further. Whatever level of confidence you have, you can always
        have more.
        \item Filling yourself up with motivation is great yet motivation
        must ultimately come from within.
        \item People can think positively and when we empower them with tools on
        how to get their goals and solve their problems, this is much more effective.
        \item It is vital to remember that positive thinking without action is useless.
        \item If you only get hyped up and inspired without taking action to change your life, it’s
        no different from someone who is not motivated at all. It’s not true motivation until
        you take action.
        \item The problem with affirmations in a vacuum is that you won’t
        gain the full results. Couple affirmations with the techniques presented in this book and you have a
        recipe for unlocking the unstoppable you.
        \item An important
        thing about affirmations is that they do not work as a stand-alone tool. Chanting some mantra over
        and over again helps drill the belief into your unconscious mind, and while that is helpful and useful
        in some contexts that alone will not get you where you want to go
        \item 
    \end{itemize}
\end{document}