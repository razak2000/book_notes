%%
%% Author: abdul
%% 8/30/18
%%

% Preamble
\chapter*{Chapters 1 to 4}
% Document
\begin{document}

    \begin{itemize}
        \item I've taken the time and space to talk about observation
        because it is one of the things important to training your
        memory.
        \item The other, and more important thing, is association.
        \item After something is observed, either by sight or
        hearing, it must, in order to be remembered, be associated
        in our minds with, or to, something we already know or re-
        member
        \item Anything you manage to remember, or have man-
        aged to remember, is only due to the fact that you have
        subconsciously associated it to something else.
        \item simple conscious associations helps
        you memorize abstract information
        \item principles and ideas of simple conscious associations be applied to remembering anything.
    \end{itemize}

    \section*{Habit is memory}
        \begin{itemize}
            \item All the things we attribute to habit, should be
            attributed to memory. Habit is memory.
            \item Mnemonics, which is a large part of a trained memory, is
            not a new or strange thing.
            \item There is no limit to the capacity of the memory.
            \item I believe that the more you remember, the more you can
            remember.
            \item The memory, in many ways, is like a muscle. A
            muscle must be exercised and developed in order to give
            proper service and use; so must the memory.
            \item You can be taught to have a
            trained memory just as you can be taught anything else.
            As a matter of fact, it is much easier to attain a trained
            memory than, say, to learn to play a musical instrument.
            \item But, I repeat, there is no such thing as a bad memory.
            There are only trained or untrained memories.
            \item Almost all
            untrained memories are one-sided.
            \item You see, most of the things you have ever remembered,
            have been associated subconsciously with something else
            that you already knew or remembered.
            \item What you subconsciously associ-
            ated strongly, will be remembered, what was not associated
            strongly, will be forgotten.
            \item There is a very thin line between a trained mem-
            ory and the true memory, and as you continue to use the
            system taught here, that line will begin to fade.
            \item That is the wonderful part about the whole thing; after
            using my system consciously for a while—it becomes auto-
            matic and you almost start doing it subconsciously
        \end{itemize}

    \section*{Test Your Memory}
        \begin{itemize}
            \item In a previous chapter, I
            gave you a few examples showing how conscious associa-
            tions are a great help in remembering anything.
            \item The more important fact that you can retain
            these simple associations over a period of years, proves it
            still more.
            \item It is my contention that if you can remember or retain
            one thing with the aid of a conscious association, you can
            do it with anything else.
        \end{itemize}

    \section*{Interest in Memory}
        \begin{itemize}
            \item Memory and observation do go hand in
            hand.
            \item You cannot possibly remember anything you do not
            observe; and it is extremely difficult to observe or remember
            anything that you do not want to remember, or that you
            are not interested in remembering.
            \item This, of course, leads to an obvious memory rule. If you
            want to improve your memory immediately, force yourself
            to want to remember.
            \item Force yourself to be interested
            enough to observe anything you want to remember or retain.
            \item I say, "force yourself," because at first a little effort
            may be necessary; however in an amazingly short time,
            you'll find that there is no effort at all required to make
            yourself want to remember anything.
            \item Aside from intending to remember, confidence that you
            will remember is also helpful.
            \item You cannot forget anything you ever really remembered.
            \item If you were to write things down with the intent of aiding
            your memory, or with the conscious thought of helping you
            to be exact with the information, that would be fine.
            \item However, using pencil and paper as a substitute for memory
            (which most people do), is certainly not going to improve
            it.
            \item You see, you usually write
            things down only because you refuse or are too lazy to take
            the slight effort or time to remember.
            \item Please keep in mind that the memory likes to be trusted.
            The more you trust it the more reliable and useful it will
            become.
            \item Writing everything down on paper without trying
            to remember, is going against all the basic rules for a
            stronger and better memory.
            \item You're not trusting your mem-
            ory; you haven't the confidence in your memory; you're not
            exercising the memory, and your interest is not strong
            enough to retain it, if you must write it down.
            \item Seriously, if you are interested in remembering, if you
            have confidence that you will remember, you have no need
            to write everything down.
            \item The important
            thing, the thing that I have been trying to stress in this
            chapter, is that interest is of great importance to memory.
            \item The thing to do is to make up your mind that you will
            be interested in remembering names, faces, dates, figures,
            facts—anything; and that you will have confidence in your
            ability to retain them.
        \end{itemize}

    \section*{Link Method of Memory}
        \begin{itemize}
            \item Before going into the actual memorizing, I must explain
            that your trained memory will be based almost entirely on
            mental pictures or images.
            \item These mental pictures will be
            easily recalled if they are made as ridiculous as you can
            possibly make them.
            \item I've told you that your trained memory will consist mostly of ridiculous mental images,
            \item Here are the twenty items that you
            will be able to memorize in sequence in a surprisingly short
            time.
                \subitem carpet, paper, bottle, bed, fish, chair, window, telephone,
            cigarette, nail, typewriter, shoe, microphone, pen, television set,
            plate, donut, car, coffee pot, and brick.
            \item I have already told you that in order
            to remember anything, it must be associated in some way to
            something you already know or remember.
            \item The first thing you have to do is to get a picture of the
            first item, "carpet," in your mind. You all know what a
            carpet is—so just "see" it in your mind's eye.
            \item he thing that you
            now know or already remember is the item, "carpet." The
            new thing, the thing you want to remember will be the
            second item, "paper."
            \item Now then, here is your first and most important step
            towards your trained memory.
            \item You must now associate or
            link carpet to, or with, paper. The association must be as
            ridiculous as possible.
            \item A sheet of paper lying on a carpet would not
            make a good association. It is too logical! Your mental pic-
            ture must be ridiculous or illogical.
            \item Take my word for the
            fact that if your association is a logical one, you will not
            remember it.
            \item You must actually see this ridiculous picture in your mind for a fraction of a second.
            \item As soon as
            you see it, stop thinking about it and go on to your next
            step.
            \item The thing you now already know or remember is,
            "paper," therefore the next step is to associate or Link,
            paper to the next item on the list, which is, "bottle."
            \item At this
            point, you pay no attention to "carpet" any longer.
            \item Make
            an entirely new ridiculous mental picture with, or between
            bottle and paper.
            \item Pick the association which
            you think is most ridiculous and see it in your mind's eye
            for a moment.
            \item I cannot stress, too much, the necessity of actually see-
            ing this picture in your mind's eye, and making the mental
            image as ridiculous as possible.
            \item You are not, however, to
            stop and think for fifteen minutes to find the most illogical
            association; the first ridiculous one that comes to mind is
            usually the best to use.
            \item We have already linked carpet to paper, and then paper
            to bottle. We now come to the next item which is, "bed."
            \item You must make a ridiculous association between bottle and
            bed. A bottle lying on a bed, or anything like that would be
            too logical.
            \item See these pictures in your mind for a mo-
            ment, then stop thinking of it.
            \item If you have actually "seen" these mental pictures in your mind's eye, you will have no trouble remem-
            bering the twenty items in sequence, from "carpet" to
            "brick."
            \item Each mental association
            must be seen for just the smallest fraction of a second,
            before going on to the next one.
            \item Why not try making your own list of objects and memorizing them in the way that
            you have just learned.
            \item I realize, of course, that we have all been brought up to
            think logically, and here I am, telling you to make illogical
            or ridiculous pictures. I know that with some of you, this
            may be a bit of a problem, at first.
            \item You may have a little
            difficulty in making those ridiculous pictures.
            \item However,
            after doing it for just a little while, the first picture that
            comes to mind will be a ridiculous or illogical one.
            \item Until
            that happens, here are four simple rules to help you.
                \subitem Picture your items out of proportion. In other words,
            too large.
                \subitem Picture your items in action whenever possible. Unfor-
            tunately, it is the violent and embarrassing things that we
            all remember; much more so than the pleasant things. So get these actions into
            your association whenever you can.
                \subitem Exaggerate the amount of items. In my sample asso-
            ciation between telephone and cigarette, I told you that you
            might see millions of cigarettes flying out of the mouth-
            piece, and hitting you in the face.
                \subitem Substitute your items. This is the one that I, per-
            sonally, use most often. It is simply picturing one item
            instead of another, i.e. Smoking a nail instead of a cigarette.
            \item 1. Out of Proportion. 2. Action. 3. Exaggeration. 4. Sub-
            stitution. Try to get one or more of these into your pictures,
            and with a little practice you'll find that a ridiculous asso-
            ciation for any two items will come to mind instantly.
            \item The
            objects to be remembered are actually linked one to the
            other, forming a chain, and that is why I call this the Link
            method of remembering.
            \item The entire Link method boils
            down to this:—Associate the first item to the second, the
            second to the third, third to the fourth, and so on.
            \item Make
            your associations as ridiculous and/or illogical as possible,
            and most important, SEE the pictures in your mind's eye.
            \item The Link system is also used
            to help memorize long digit numbers and many other
            things. However, don't jump ahead of yourself; don't
            worry about those things now
            \item Of course, you can use the Link immediately to help
            you remember shopping lists, or to showoff for your friends.
            \item If when you try this you find that you
            are having trouble recalling the first item, I suggest that you
            associate that item to the person that's testing you.
            \item Also, if on first trying
            this as a stunt, you do forget one of the items, ask what it
            is and strengthen that particular association.
            \item You either
            didn't use a ridiculous enough association, or you didn't
            see it in your mind, or you would not have forgotten it.
            \item The most impressive part of it, is that if your friend asks
            you to call off the items two or three hours later, you will
            be able to do it! They will still be brought to mind by
            your original associations
        \end{itemize}

    \section*{Peg System of Memory}
        \begin{itemize}
            \item most people would
            say it is impossible to remember so many jokes by number.
            \item First, however, you must learn
            how to remember the numbers.
            \item Numbers themselves are
            about the most difficult things to remember because they
            are completely abstract and intangible. It is almost impossi-
            ble to picture a number.
            \item The problem is to be able to
            associate any and all numbers easily, quickly, and at any
            time.
            \item Anything you wish to remember from now on, having to do
            with numbers in any way, you will be able to "hang" on
            these pegs! That is why I call this the PEG system of
            memory.
            \item The PEG system will show you how to count with
            objects (which can be pictured) instead of numbers.
            \item In order for you to learn the method, you must first
            learn a simple phonetic alphabet.
                \subitem The sound for \#1 will always be—T or D. The letter
            T has one downstroke.
                \subitem The sound for \#2 will always be—N. Typewritten n has
            two downstrokes.
                \subitem The sound for \#3 will always be—M. Typewritten m
            has three downstrokes.
                \subitem The sound for \#4 will always be—R. Final sound of
            the word, "four" is R.
                \subitem The sound for \#5 will always be—L. Roman numeral
            for 50 is L.
                \subitem The sound for \#6 will always be— ch, sh, soft g, etc.
                \subitem The sound for \#7 will always be—K, hard c, hard g. The
number 7 can be used to form a K. One seven rightside up,
and the other upside down.
                \subitem The sound for \#8 will always be—F or V. Written f
and figure 8 both have two loops, one above the other.
                \subitem The sound for \#9 will always be—P or B. The number
9. turned around is P.
                \subitem The sound for \#0 (zero) will always be—S or Z. First
sound of the word, "zero."

            \item Please keep in mind that the letters are not important; we are interested in the sound only.
            \item TeN
            MoRe LoGiC FiBS. This will help you to memorize the
            sounds in order from one to zero.
            \item It is necessary, however, to know them out of sequence
            \item This simple phonetic alphabet is of utmost importance,
            and the sounds should be practiced until they are second
            nature to you.
            \item Here is a method of practice to help you learn the sounds thoroughly:— Anytime you see
            a number, break it down into sounds in your mind.
            \item None of the vowels, a e i o or u have any meaning at all
            in the phonetic alphabet; neither do the letters w, h or y.
            (Remember the word, "why").
            \item You are ready now to learn some of those "pegs" I men-
            tioned. I would suggest however, that you know the sounds
            thoroughly before you go on to the pegs themselves
            \item Each one of the peg words that I will give you has
            been specially chosen because it is comparatively easy to
            picture in your mind, and that is important.
                \subitem From here on in, the word, "tie"
            will always represent the number 1 to you.
                \subitem The word, "NOAH" will always represent \#2. Picture
            an old, white haired man on an ark.
                \subitem The word, "MA" will always mean \#3. Here I suggest
            that you always picture your own mother.
                \subitem The word, "RYE" will always represent the number 4.
            You can picture either a bottle of Rye whiskey or a loaf of
            rye bread.
                \subitem The word, "LAW" will always represent \#5. The word
            "law" itself, cannot be pictured; I suggest that you picture
            any policeman, in uniform, because they represent the law.
                \subitem Number 6 is the word "SHOE."
                \subitem Number 7 is the word
            "COW."
                \subitem Number 8 is the word, "IVY." For this one, you
            can picture either Poison Ivy, or ivy growing all over the sides of a house.
                \subitem Number 9 is the word, "BEE."
                \subitem Number 10 has two digits, the digit 1 and a zero. The peg word
            for \#10 therefore must be made up of a t or d sound and
            an s or z sound, in that order. We'll use the word, "TOES"
            —picture your own toes.
                \subitem 11. tot. For "tot," it is best to picture a child that you know.
                \subitem 12. tin. For \#12, you can see the object called, made out of "tin."
                \subitem 13. tomb. For "tomb," picture a gravestone.
                \subitem 14. tire
                \subitem 15. towel
                \subitem 16. dish
                \subitem 17. tack
                \subitem 18. dove
                \subitem 19. tub
                \subitem 20. nose. For \#20, you can see
            the object called, on your face in place of your "nose."
                \subitem 21. net. For
            "net," you can use either a fishing net, a hair net, or a
            tennis net.
                \subitem 22. nun
                \subitem 23. name. For \#23, you can see the object you wish to remember
            forming your "name."
                \subitem 24. Nero
                \subitem 25. nail
            \item Once you decide on a particular mind picture for
            this, or for any of the pegs, use that particular picture
            always.
            \item The first illogical picture that
            comes to mind is usually the best one to use, because that is the
            one that will come to mind later on.
            \item Whatever you
            decide on, you must use it all the time.
            \item Remember, please, that once you decide on a particular
            picture for any of the peg words, you are to use that picture
            all the time.
        \end{itemize}

    \section*{Uses of the Peg and Link Systems}
        \begin{itemize}
            \item The beautiful thing about the Link method is that
            you can forget a list whenever you wish.
            \item Actually, when
            you memorize the second shopping list, the first one fades
            away.
            \item You can, of course, retain as many lists or links as
            you desire.
            \item If you have memorized a list of items
            with the Link system, which you want to retain—you can.
            If you want to forget the list—you can. It is merely a ques-
            tion of desire.
            \item If it happens to be a list that
            you do not intend to utilize right away, but which you feel
            you want to retain for future use—you can do that, too.
            \item You would have to go over the list in your mind the day
            after you memorized it. Then go over it again a few days
            later.
            \item After doing this a few times, you have filed the list
            away, and it will be ready when you need it.
            \item The main difference between the Link and the Peg
            methods is that the Link is used to remember anything in
            sequence, while the Peg is for memorizing things in and out
            of order.
            \item You can use
            either the Peg or Link methods, or one, in conjunction with
            the other.
            \item Before completing this chapter, please learn the pegs for
            \#26 through to \#50.
                \subitem 26. notch
                \subitem 32. moon
                \subitem 38. movie
                \subitem 44. rower
                \subitem 27. neck
                \subitem 33. mummy
                \subitem 39. mop
                \subitem 45. roll
                \subitem 28. knife
                \subitem 34. mower
                \subitem 40. rose
                \subitem 46. roach
                \subitem 29. knob
                \subitem 35. mule
                \subitem 41. rod
                \subitem 47. rock
                \subitem 30. mice
                \subitem 36. match
                \subitem 42. rain
                \subitem 48. roof
                \subitem 31. mat
                \subitem 37. mug
                \subitem 43. ram
                \subitem 49. rope
                \subitem 50. lace
            \item A good way to practice this would be to remember a list
            of twenty-five objects, in and out of sequence, using the
            peg words from 26 to 50 to do it. Just number the paper
            from 26 to 50 instead of 1 to 25.
            \item After a day or so, if you
            feel ambitious, you can try a list of fifty items.
            \item
        \end{itemize}

    \section*{How to Train Your Observation}
        \begin{itemize}
            \item As I said earlier in the book, although my systems actually
            force you to observe if you apply them—your sense of
            observation can be strengthened with a little practice.
            \item If
            you're interested in helping your memory, don't sell observa-
            tion short.
            \item You just can't remember anything that you do not observe to begin with.
            \item If you haven't observed, then you haven't realized, and
            what you haven't realized you can't forget, since you never
            really remembered it in the first place.
            \item If you want to take the time, it is a simple matter to
            strengthen your sense of observation.
                \subitem Take a
            piece of paper, and without looking around you, list every-
            thing in the room
                \subitem Don't leave out anything you can think
            of, and try to describe the entire room in detail. List every
            ashtray, every piece of furniture, pictures, doodads, etc.
                \subitem Now, look around the room and check your list.
                \subitem Notice all
            the things you did not put down on your list, or never
            really observed, although you have seen them countless
            times. Observe them now!
                \subitem Step out of the room and test
            yourself once more. Your list should be longer this time.
                \subitem If you keep at this, your observation will be keener
            no matter where you happen to be.
            \item Of course, you can't go around looking for scenes
            to observe—but, you can practice in this way:
                \subitem Think of
            someone whom you know very well. Try to picture his or
            her face; now see if you can describe the face on paper.
                \subitem List everything you can possibly remember. Go into detail
            —list color of hair and eyes, complexion, any or all out-
            standing features, whether or not they wear glasses, what
            type of glasses, type of nose, ears, eyes, mouth, forehead,
            approximate height and weight, hairline, on which side is
            the hair parted, is it parted at all, etc., etc.
                \subitem The next time
            you see this person, check yourself.
                \subitem Note the things you
            didn't observe and those you observed incorrectly. Then
            try it again! You will improve rapidly.
                \subitem A good way to practice this is in a subway or bus, or any
            public conveyance.
                \subitem Look at one person for a moment, close
            your eyes and try to mentally describe every detail of this
            person's face. Pretend that you are a witness at a criminal
            investigation, and your description is of utmost importance.
                \subitem Then look at the person again (don't stare, or you will be
            in a criminal investigation) and check yourself.
                \subitem You'll find
            your observation getting finer each time you try it.

            \item One last suggestion as to a form of practice.
                \subitem Look at any
            shop window display.
                \subitem Try to observe everything in it (with-
            out using the Peg or Link systems).
                \subitem Then list all the items
            without looking at the display.
                \subitem You can wait until you're
            home to do this; then go back to check, when you can.
                \subitem Note
            the items you left out and try it again
                \subitem When you think
            you've become proficient at it, try remembering the prices
            of the items also.
            \item Each time you do any of these exercises, your sense of
            observation will become noticeably sharper.
            \item Although all
            this is not absolutely necessary for the acquiring of a trained
            memory, it is a simple matter to strengthen your observa-
            tion.
            \item If you take the little time to practice, you will soon
            begin to observe better, automatically.
            \item Before reading any further, I would suggest that youubitem You could, of course, make up your own words, as long
            as they stay in the phonetic alphabet system. These would
            probably serve you just as well, but you might pick some
            words that would conflict with some of the words that
            you will eventually learn for other purposes. So, wait until
            you've finished the book, and then change words to your
            heart's content.
                \subitem 51. lot
                \subitem 52. lion
                \subitem 53. loom
                \subitem 54. lure
                \subitem 55. lily
                \subitem 56 leech
                \subitem 57. log
                \subitem 58. lava
                \subitem 59. lip
                \subitem 60. cheese
                \subitem 61. sheet
                \subitem 62. chain
                \subitem 63. chum
                \subitem 64. cherry
                \subitem 65. jail
                \subitem 66. choo choo
                \subitem 67. chalk
                \subitem 65. chef
                \subitem 75. coal
                \subitem 69. ship
                \subitem 70. case
                \subitem 71. cot
                \subitem 72. coin
                \subitem 73. comb
                \subitem 74. car
                \subitem 75. coal
            \item I would suggest that you
            memorize the Peg Words from 51 to 75
            \item  I might also suggest that for the time being, you use the words that I give
            you.

        \end{itemize}

    \section*{It Pays to Remember Speeches, Articles,
    Scripts and Anecdotes}
        \begin{itemize}
            \item If a speech is
            memorized word for word, and then a word, here and there,
            is forgotten; it surely will not be delivered as it should be.
            \item Even if you are well versed in your subject, you
            may forget some of the facts you want to speak about.
            \item I believe that the best way to prepare a speech is to lay
            it out thought for thought. Many of our better speakers do
            just that.
            \item They simply make a list of each idea or thought
            that they want to talk about, and use this list in lieu of
            notes.
            \item In this way, you can't forget words, since yon haven't
            memorized any. You can hardly lose your place; one glance
            at your list will show you the next thought to put into
            words
            \item But, for those of you who would rather not rely on pieces
            of paper—the Link method can help you easily.
            \item If you wish
            to memorize your speech thought for thought, from the
            beginning to the end, you would be forming a sequence.
            That's why you would use the Link method of memory to
            memorize it.
            \item "If you remember
            the main, the incidentals will fall into place.". You actually
            never forget anything you've remembered, you just have to
            be reminded of it. So, if you remember the main thoughts of your speech,
            the incidentals, the ifs, ands and buts, will fall into place.
            \item The same ideas are used to memorize any article you
            read, if you desire.
                \subitem First read the article, of course, to get
            the "gist" of it.
                \subitem Then pick out the Key Words for each
            thought; then make a link to remember them, and you've
            got it.
                \subitem With a bit of practice, you'll actually be able to do
            this as you read.
            \item Many times while reading for enjoyment, I'll come across
            some piece of information that I'd like to remember. I
            simply make a conscious association of it, while I'm read-
            ing. This idea can, if used enough, speed up your reading
            considerably.
            \item There is no need to associate everything; just the points that you feel are necessary to remember.
            \item The same system of linking Key Words can be used
            for remembering lyrics and scripts. Of course, in this case
            it is usually necessary to memorize them word for word.
            \item If
            you have trouble memorizing your cues in a play, why not
            associate the last word of the other actor's line to the first
            word of your line.
            \item Your memory for stories and anecdotes will improve
            immediately if you use the Key Word system.
            \item Just take one
            word from the story, a word from the punch line is usually
            best, that will bring the entire joke to mind.
            \item When you
            get your Key Words, you can either link them to each other to remember all the stories in sequence, or use the Peg sys-
            tem to remember them by number.
            \item To memorize the
            pages of any picture magazine, all you have to do is to asso-
            ciate the peg word that represents the page number to the
            highlight of that page.
            \item As you use my systems you'll find your
            true memory getting stronger. The best example of this is
            in memorizing a magazine.
            \item In order to make the associa-
            tions in the first place, you must really see and observe the
            picture on the page.
            \item Before reading any further, learn the last of the one
            hundred peg words.
                \subitem 76. cage
                \subitem 77. coke
                \subitem 78. cave
                \subitem 79. cob
                \subitem 80. fez
                \subitem 81. fit
                \subitem 82. phone
                \subitem 83. foam
                \subitem 84. fur
                \subitem 85. file
                \subitem 86. fish
                \subitem 87. fog
                \subitem 88. fife
                \subitem 89. fob
                \subitem 90. bus
                \subitem 91. bat
                \subitem 92. bone
                \subitem 93. bum
                \subitem 94. bear
                \subitem 95. bell
                \subitem 96. beach
                \subitem 97. book
                \subitem 98. puff
                \subitem 99. pipe
                \subitem 100. thesis or disease
            \item After learning these, you should be able to count from
            one to one hundred quickly, with your peg words only.
            \item The beauty of it is that you don't have to take time out
            to practice them. If you're traveling to work, or doing any-
            thing that doesn't require thought—you can go over all the
            pegs in your mind.
            \item If you go over them just every once in
            a while, they'll soon be as familiar to you as the numbers
            from one to one hundred.

        \end{itemize}

    \section*{It Pays to Remember Playing Cards}
        \begin{itemize}
            \item The memory stunts you will do with cards after studying
            these methods will seem almost as amazing to your friends.
            Aside from that, they are also wonderful memory exercises.
            \item I suggest that you read and learn the contents of this chap-
            ter whether or not you indulge in card playing.
            \item There are actually two things that you must know in
            order to remember cards.
                \subitem First, is a list of at least fifty-two
            peg words for the numbers 1 to 52;
                \subitem You also have to know a peg word for every card in a deck
            of cards.
                \subitem These card peg words are not chosen at random.
                \subitem As with the number pegs, they are selected because they are easy to picture, and because they follow a definite sys-
            tem.
            \item Barring a few exceptions which will be discussed later,
            every card peg word will begin with the initial letter of the
            card suit
            \item Each word will end with a consonant sound; this sound will represent the numerical value
            of the card, according to our phonetic alphabet.
            \item Here are all fifty-two card peg words. Look them over
            carefully, and I assure you that you can know and retain
            them with no more than perhaps twenty minutes to a half
            hour of study.
            \item \textcolor{red}{--- FINISH ME -----}
        \end{itemize}

    \section*{It Pays to Remember Long Digit Numbers}
        \begin{itemize}
            \item numbers can be remembered if they are made to represent
            or mean something to us
            \item The Peg system of memorizing long digit numbers is
            actually a combination of the Peg and the Link methods.
            \item In
            actual practice, you should form your peg words and link
            them as you move your eyes from left to right across the
            number.
            \item If you like, you can link only
            four words in order to memorize a twelve digit number
            \item Just make up words to fit three of the digits at a time, and
            link those.
            \item If a long digit number that you wish to remember falls
            into line for words that fit four digits at a time—why not
            use them! In that way you can sometimes memorize and
            retain a twenty digit number, by linking only five words
            \item If, in your particular business, you find it necessary to
            memorize long numbers very often, you'll soon use the first
            word that pops into your mind to fit either the first two,
            three or four digits.
            \item There is no rule that says you must use
            words to fit the same amount of digits in any long digit
            number.
            \item To memorize the number quickly, you use any
            words at all—usually you will have time to think for a mo-
            ment to find the best words for the number to be memo-
            rized.
        \end{itemize}

    \section*{Some Pegs for Emergencies}
        \begin{itemize}
            \item I have had occasion to need a few
            short peg lists to help me recall up to twenty or twenty-six
            items.
            \item Well, there are two methods that I've used quite
            often. The first is to use the twenty-six letters of the alpha-
            bet. All you have to do is to make up a word for each letter
            which sounds like the letter itself.
                \subitem A — ape
                \subitem B — bean
                \subitem C — sea
                \subitem D — dean
                \subitem E — eel
                \subitem F — effort (or effervescent)
                \subitem G — jean (or Gee,command to horse)
                \subitem H — ache
                \subitem I — eye
                \subitem J — jail
                \subitem K — cake
                \subitem L — el (elevated train)
                \subitem M — ham
                \subitem N — hen
                \subitem O — eau (water)
                \subitem P — pea
                \subitem Q — cute
                \subitem R — hour (clock)
                \subitem S — ass
                \subitem T — tea
                \subitem U — ewe
                \subitem V — veal
                \subitem W — Waterloo
                \subitem X — eggs
                \subitem Y — wine
                \subitem Z — zebra
            \item If you go over this list once or twice, you'll have it. De-
            cide on a picture for each one, and use that all the time.
            \item Of course, there are other words that can fit for some letters,
            and you can use any that you like. Just be sure that they
            do not conflict with your basic list of pegs. The words listed
            above are the ones that I use.
            \item Incidentally, if you made a link from zebra to ape, you
            would be able to recite the alphabet backwards, which is
            quite a feat in itself.
            \item If you want to, you can associate each
            letter word to your regular peg word for that particular num-
            ber.
            \item In this way you would know the numerical position of
            each letter immediately:—ape to "tie"; bean to "Noah";
            sea to "ma," and dean to "rye," etc
            \item Another idea I use is to make a list of nouns, each of
            which look like the number they represent.
                \subitem For \#1, you
            might picture a pencil, because a pencil standing upright
            looks like the numeral one.
                \subitem For \#2, you can picture a swan;
            a swan on a lake is shaped something like the numeral
            two.
                \subitem A table
            or chair, or anything with four legs can represent \#4
            \item A bat and ball pictured side by side can represent
            \#10;
            \item I picture spaghetti for \#11; my original picture was of two
            pieces of raw spaghetti lying side by side, which looked like
            \#11
            \item For \#12, you can think of 12:00 o'clock and picture
            a clock.
            \item Your basic list of peg words can
            be brought up to a thousand, or over, if you wanted to, and
            the beauty of it is that as soon as you heard one of them,
            the sounds in the word would tell you immediately which
            number it represented.
        \end{itemize}

    \section*{It Pays to Remember Dates}
        \begin{itemize}
            \item I have come across a very simple way
            to find the day of the week for any date of the current
            year.
            \item All you have to do is memorize the twelve digit num-
            ber of the first Sunday of each month of the current year
            \item the memory feat that follows is also a
            practical method of knowing the day of the week for any
            date in the twentieth century.
            \item To accomplish this you must know two things besides
            the month, day and year:
                \subitem a certain number for the year,
                which I will refer to as the "year key,"
                \subitem a certain num-
                ber for the month, which I'll call the "month key."
            \item The keys for the
            years and the months are all either 0, 1, 2, 3, 4, 5, or 6.
            \item Sevens are always removed as soon as possible.
            \item These are the month keys, which will always remain the
            same:
                \subitem January — 1
                \subitem February — 4
                \subitem March — 4
                \subitem April — 0
                \subitem May — 2
                \subitem June — 5
                \subitem July — 0
                \subitem August — 3
                \subitem September — 6
                \subitem October — 1
                \subitem November — 4
                \subitem December — 6
            \item Just associate the peg word for the
            last two digits of the year, to the word that you are using to
            represent the key numbers.
                \subitem 0 - 1900, 1906, 1917, 1923, 1928, 1934, 1945, 1951, 1956, 1962, 1973, 1979, 1984
                \subitem 1 - 1901, 1907, 1912, 1918, 1929, 1935, 1940, 1946, 1957, 1963, 1968, 1974, 1985
                \subitem 2 - 1902, 1913, 1919, 1924, 1930, 1941, 1947, 1952, 1958, 1969, 1975, 1980, 1986
                \subitem 3 - 1903, 1908, 1914, 1925, 1931, 1936, 1942, 1953, 1959, 1964, 1970, 1976, 1981, 1987
                \subitem 4 - 1909, 191;, 1920, 1926, 1937, 1943, 1948, 1954, 1965, 1971, 1982
                \subitem 5 - 1904, 1910, 1921, 1927, 1932, 1938, 1949, 1955, 1960, 1966, 1977, 1983
                \subitem 6 - 1905, 1911, 1916, 1922, 1933, 1939, 1944, 1950, 1961, 1967, 1972, 1978
            \item You now have all the information necessary to do the cal-
            endar stunt, except for one thing. If it is a leap year and the
            date you are interested in is for either January or February
            —then the day of the week will be one day earlier than
            what your calculations tell you.
            \item You can tell if a year is a leap year by
            dividing four into the last two digits. If four goes in evenly,
            with no remainder, then it is a leap year.
            \item The year 1900 is not a leap year.
        \end{itemize}

    \section*{It Pays to Remember Foreign Language
    Vocabulary and Abstract Information}
        \begin{itemize}
            \item Perhaps the best example is in trying to memorize for-
            eign language vocabulary
            \item A word in a foreign language is
            nothing but a conglomeration of sounds to anyone who is not familiar with the language. That's why they're so diffi-
            cult to remember.
            \item To make them easier to remember you will use the sys-
            tem of SUBSTITUTE WORDS.
            \item Substitute words or
            thoughts are used whenever you want to remember anything that is abstract, intangible or unintelligible; something that makes no sense to you, can't be pictured, yet
            must be remembered.
            \item Making up a substitute word is simply this:— Upon
            coming across a word that means nothing to you; that is in-
            tangible and unintelligible, you merely find a word, phrase
            or thought that sounds as close to it as possible, and that is
            tangible and can be pictured in your mind.
            \item Any word you may have to remember, foreign language
            or otherwise, that is meaningless, can be made to mean
            something to you by utilizing a substitute word or thought.
            \item The thing for you to keep in
            mind is that the thought or picture that comes to you when
            you hear any intangible word, is the one to use.
            \item The substitute word you select
            does not have to sound exactly like the foreign word you're
            trying to remember.
            \item As long as you have the main part of the word in your picture, the
            incidentals, the rest of the word, will fall into place by true
            memory.
            \item To remember a foreign word and its English meaning, associate the
            English meaning to your substitute word for the foreign
            word.
            \item Try this method with any foreign language vocabulary,
            and you'll be able to memorize the words better and faster,
            and with more retentiveness, than you ever could before.
            \item Aside from languages, this system can be used for anything
            you may be studying which entails remembering words
            that have no meaning to you, at first.
            \item The point is that the substitute word or thought has
            meaning while the original word does not.
            \item Therefore you
            make it easier to remember by using the substitute word.
        \end{itemize}

    \section* {\textcolor{blue}{It Pays to Remember Names and Faces}}
    \section* {\textcolor{blue}{What's in a Name?}}
    \section* {\textcolor{blue}{More about Names and Faces}}
    \section* {\textcolor{blue}{It Pays to Remember Facts about People}}
    \section* {\textcolor{blue}{It Pays to Remember Telephone Numbers}}
    \section* {The Importance of Memory}
        \begin{itemize}
            \item After completing this book, I hope that all of you will be
            able to remember anything you read, that is, if you want to.
            \item As I've mentioned before, you can remember anything if
            you so desire. These memory systems just make it easier for
            you.
            \item Most everything we do is based on
            memory. The things we often say we do by "instinct," are
            really done through memory.
            \item Writing things down just isn't enough in itself to help
            you remember.
            \item It is getting over the first hurdle that is always the most
            difficult in any new thing you learn.
            \item The first hurdle in
            training your memory, is t
            \item o actually use my system. Use
            it, and it'll work for you.
            \item Just knowing the system and still
            writing phone numbers on paper, is the same as not knowing the system at all.
            \item The thing to keep in mind, above all else, is to make all
            your associations ridiculous and/or illogical.
            \item Many of the
            systems being taught today, and those in the past, do not
            stress this nearly enough.
            \item
        \end{itemize}
    \section* {Don't Be Absent-minded}
        \begin{itemize}
            \item Many people make the mistake of confusing absent-mindedness with a poor memory.
            \item People
            with excellent memories can also be absent-minded
            \item I believe that you can cure absentmindedness with just a
            little effort
            \item Actually, absentmindedncss is nothing more than inattention
            \item The little things
            that we do continually, like putting down things, are just
            not important enough to occupy our minds—so, we become
            absent-minded.
            \item To avoid absentmindedness,
            think what you're doing.
            \item Okay, then, why not use conscious associations to help you
            remember trivial things? You can, you know, and it's easy
            to do
            \item First decide what it is that you do or see at the very
            last moment upon leaving your house.
            \item I personally see the
            doorknob of my front door, because I check it to see if the
            door is locked.
                \subitem That is the last thing I do, so I make a ridicu-
                lous association between doorknob and letter.
                \subitem When I leave
                my house the next morning, I'll check the doorknob; once
                I think of doorknob, I'll recall my ridiculous association
                and remember that I must take the letter!
            \item Now, how can you be sure to mail the letter?
            \subitem make an association
            between the person the letter is going to, and the mail box.
            \subitem When you see a mail box, in
            the street, it will remind you to mail the letter
            \item This idea can be used for all the little things you want
            to remember to do.
            \item Now we come to the real petty annoyances of absent-
            mindedness; such as putting things down, and then forget-
            ting where they are.
            \item You have to make an association between
            the object and its location.
            \item The same thing
            goes for any small item or small errand. If you're in the
            habit of putting things down anyplace, get into the habit
            of making an association to remind you of where it is.
            \item One of the questions usually asked at this point is:—
            "Fine, but how am I going to remember to make these as-
            sociations for all these petty things?" There is only one an-
            swer to this question—use some will power at first, and
            be sure that you do make the associations.
            \item the eyes cannot
            see if the mind is absent—and your mind is absent when
            you put things away mechanically. The very idea of making
            an association makes you think of what you're doing for at
            least a fraction of a second, and that's all that's necessary.
            \item Just as absentmindedness is often mistaken for a poor
            memory; so is absentmindedness often blamed for mental
            blocks.
            \item The mind must work in its own devious way; more
            often than not, just thinking around the fact you want, will
            make it pop into your mind.
            \item If this doesn't help, the next best thing is to forget about
            it. Stop thinking about it completely for awhile, and the
            odds are it will come to you when you least expect it.
        \end{itemize}
    \section* {\textcolor{blue}{Amaze Your Friends}}
    \section* {\textcolor{blue}{It Pays to Remember Appointments and Schedules}}
    \section* {\textcolor{blue}{It Pays to Remember Anniversaries, Birthdays and Other Important Dates}}
    \section* {\textcolor{blue}{Memory Demonstrations}}
    \section* {\textcolor{blue}{Use the Systems}}


\end{document}

































