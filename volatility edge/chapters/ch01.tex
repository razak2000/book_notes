%%
%% Author: abdul
%% 8/2/18
%%
\chapter{ Introduction }

% Document
\begin{document}

    \begin{itemize}
        \item Introduction - Dow Jones Industrial Average (DJIA), massive collapse, predict the drawdown, cross-market trading, circuit-breaker procedure; stock, options, and index futures markets; velocity of price movements, U.S. corporate earnings, economic slowdown and deflationary pressures, market turmoil, equity exposure, October 1997 drawdown, bubbles, credit crisis,  importance of balanced trade, risks implied by deficits and surpluses, variety of market conditions, short-term trends, expected return, worst-case loss, personal opinion, trading positions, underlying financial assumptions, performance of any particular company, index, or industry, interest rates,  mathematical models, world's economy, flaws in the mathematical model, complex credit derivative products, market swings, ratings downgrades, risk simulations based on historic data, U.S. gross domestic product (GDP), world's GDP, world's derivatives markets, shape of the curve, distribution of daily price changes, potential change in a stock's implied volatility,  predict short-term changes in price, Volatility, technical bottoms, likelihood of a significant market crash, fundamental mathematical properties of stocks and indexes, Options, underlying pricing models, mathematical constructs of volatility and time, strategy for dynamically managing option positions, strategy for selecting and structuring trades, Adaptive trading, precise metrics and rules, making adjustments to complex positions, bounded and closely related set of option trading strategies, inconsistencies in the models used to price options,  arbitrage opportunities, volatility index (VIX), evolving discipline, new set of market conditions, tuning of the system
        \item PRICE DISCOVERY AND MARKET STABILITY - behavior of equity markets, markets crash, stabilizing forces, typical drawdown,
        Price Discovery Is a Chaotic Process
        PRACTICAL LIMITATIONS OF TECHNICAL CHARTING
        BACKGROUND AND TERMS
        SECURING A TECHNICAL EDGE
        \item
    \end{itemize}

\end{document}
